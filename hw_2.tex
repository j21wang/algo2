\documentclass{article}
\usepackage{amssymb}
\usepackage{amsmath}
\usepackage{amsfonts}
\usepackage{latexsym}
\usepackage{times}
%\usepackage{psfrag,psfig,epsfig,epsf}
\usepackage{graphics}
\usepackage{multirow}
\usepackage{fullpage}
\usepackage{verbatim}
\usepackage{fancyheadings}
\usepackage[T1]{fontenc}
\usepackage{arev}
\usepackage{subfigure}
\usepackage{url}
\usepackage[noline,noend,ruled,linesnumbered]{algorithm2e}
\usepackage{algpseudocode}
\linespread{1.02} 

\pagestyle{empty}

\addtolength{\topmargin}{-20pt}
\addtolength{\oddsidemargin}{-5pt}
\addtolength{\textwidth}{20pt}
\addtolength{\textheight}{50pt}

\newenvironment{myitem}{\begin{list}{$\bullet$}
{\setlength{\itemsep}{-0pt}
\setlength{\topsep}{0pt}
\setlength{\labelwidth}{0pt}
%\setlength{\labelsep}{0pt}
\setlength{\leftmargin}{10pt}
\setlength{\parsep}{-0pt}
\setlength{\itemsep}{0pt}
\setlength{\partopsep}{0pt}}}%
{\end{list}}

\begin{document}

\sloppy

\noindent {\bf Joyce Wang, Matt Demusz, Frank Porco}\\


\noindent \underline{CS 344: DESIGN AND ANALYSIS OF COMPUTER
  ALGORITHMS \hspace{1.6in} SPRING 2014}

\vspace{0.1in}

\begin{center}
{\bf {\large Homework 2}}\\
Divide-and-Conquer Algorithms, Sorting Algorithms, Greedy Algorithms\\
\end{center}

\vspace{0.1in}

\noindent Deadline: March 14, 11:59pm.\\ 
Available points: 110. Perfect score: 100.\\

\begin{center}
{\bf Homework Instructions:}
\end{center}

\noindent {\bf Teams:} Homeworks should be completed by teams of
students - three at most. No additional credit will be given for
students that complete a homework individually. Please inform
Athanasios Krontiris {\bf if your team has changed from the previous
  assignment.} (email: ak979 AT cs.rutgers.edu).\\

\noindent {\bf Submission Rules:} Submit your solutions electronically
as a PDF document through \url{sakai.rutgers.edu}. Do not submit Word
documents, raw text, or hardcopies etc. Make sure to generate and
submit a PDF instead. Each team of students should submit only a
single copy of their solutions and indicate all team members on their
submission.  Failure to follow these rules will result in lower grade
in the assignment.\\

\noindent {\bf Late Submissions:} No late submission is allowed. If
you don't submit a homework on time, you get 0 points for that
homework.\\

\noindent {\bf Extra Credit for \LaTeX:} You will receive 10\% extra
credit points if you submit your answers as a typeset PDF (using
\LaTeX, in which case you should also submit electronically your
source code). Resources on how to use \LaTeX\ are available on the
course's website. There will be a 5\% bonus for electronically
prepared answers (e.g., on MS Word, etc.) that are not typeset.\\

\noindent {\bf 25\% Rule:} For any homework problem (same will hold
for exam questions), you can either attempt to answer the question, in
which case you will receive between 0 and 100\% credit for that
question (i.e., you can get partial credit), or you can write "I don't
know", in which case you receive 25\% credit for that question.
Leaving the question blank is the same as writing "I don't know." You
can and will get less than 25\% credit for a question that you answer
erroneously.\\

\noindent {\bf Handwritten Reports:} If you want to submit a
handwritten report, scan it and submit a PDF via Sakai. We will not
accept hardcopies. If you choose to submit handwritten answers and we
are not able to read them, you will not be awarded any points for the
part of the solution that is unreadable.\\

\noindent {\bf Precision:} Try to be precise. Have in mind that you
are trying to convince a very skeptical reader (and computer
scientists are the worst kind...) that your answers are correct.\\

\noindent {\bf Collusion, Plagiarism, etc.:} Each team of students
must prepare its solutions independently from other teams, i.e.,
without using common notes or worksheets with other students or trying
to solve problems in collaboration with other teams.  You must
indicate any external sources you have used in the preparation of your
solution. Do not plagiarize online sources and in general make sure
you do not violate any of the academic standards of the course, the
department or of the university (the standards are available through
the course's website). Failure to follow these rules may result in
failure in the course.\\

\newpage

\vspace{0.1in}

{\bf }

\begin{center}
{\bf Part A (20 points)}
\end{center}

\noindent {\bf Problem 1:} The more general version of the Master
Theorem is the following. Given a recurrence of the form: 
$$T(n) = a T(\frac{n}{b}) + f(n)$$
where $a \geq 1$ and $b > 1$ are constants and $f(n)$ is an
asymptotically positive function, there are 3 cases: 
\begin{enumerate}
\item If $f(n) = O(n^{log_ba - \epsilon}$) for some constant $\epsilon
  > 0$, then $T(n) = \Theta(n^{log_ba})$.
\item If $f(n) = \Theta(n^{log_ba} log^kn)$ with $k \geq 0$, then
  $T(n) = \Theta(n^{log_ba} log^{k+1}n)$. In most cases, $k = 0$.
\item If $f(n) = \Omega(n^{log_ba+\epsilon})$ with $\epsilon > 0$, and
  $f(n)$ satisfies the regularity condition, then $T(n) = \Theta( f(n)
  )$. The regularity condition specifies that $a f(\frac{n}{b}) \leq c
  f(n)$ for some constant $c < 1$ and all sufficiently large $n$.
\end{enumerate}

\noindent Give asymptotic bounds for the following recurrences. Assume
$T(n)$ is constant for $n = 1$. Make your bounds as tight as possible,
and justify your answers.\\

\noindent A. $T(n) = 2T(\frac{n}{4}) + n^{0.51}$\\
\noindent B. $T(n) = 16 T(\frac{n}{4}) + n!$\\
\noindent C. $T(n) = \sqrt{2} T(\frac{n}{2}) + logn$\\
\noindent D. $T(n) = T(n-1) + lgn$\\
\noindent E. $T(n) = 5T(n/5) + \frac{n}{lgn}$\\

\begin{center}
{\bf Part B (25 points)}
\end{center}

\noindent {\bf Problem 2:} You are in the HR department of a
technology firm, and here is a job for you.  There are $n$ different
projects, and $n$ different programmers.

Every project has its unique payoff when completed and level of
difficulty (which are uniform, regardless which programmer will work
on the project).  Every programmer has a unique skill set as well as
expectations for compensation (which are uniform, regardless the
project the programmer will work on). You cannot directly collect
information that allows you to compare the payoff or difficulty level
of two projects, or the capability or expectations for compensation of
two programmers.

Instead, you can arrange a meeting between each project manager and
programmer.  In each meeting, the project manager will give the
programmer an interview to see whether the programmer can do the
project; the programmer can ask the project manager about the
compensation to see whether it meets her expectations.  After the
meeting, you can get a result based on the feedback of the project
manager and the programmer.  The result can be:

\begin{enumerate}
  \item The programmer can't do the project.
  \item The programmer can do the project, but the compensation of the
    project doesn't meet her expectation.
  \item The programmer can do the project, and the compensation for
    the project matches her expectations. At this time, we say the
    project and the programmer \textit{match} with each other.
\end{enumerate}

Assume that the projects and programmers match one to one. Your goal
is to match each programmer to a project.

{\bf A.} Show that any algorithm for this problem must need $\Omega( n
\log{n} )$ meetings in the worst case.

{\bf B.}  Design a randomized algorithm for this problem that runs in
expected time $O ( n \log{n} )$.

\newpage

\begin{center}
{\bf Part C (40 points)}
\end{center}


\noindent {\bf Problem 3:} A nation-wide programming contest is held
at $k$ universities in North America. The $i^{th}$ university has
$m_{i}$ participants. The total number of participants is $n$, i.e.,
$n = \sum_{i=1}^{k}m_{i}$. In the contest, participants have to write
programs to solve 6 problems. Each problem contains 10 test cases,
each test case is worth 10 points. Participants aim to maximize their
collected points.  \\


After the contest, each university sorts the scores of participants
belonging to it and submits the grades to the organizer.  Then the
organizer has to collect the sorted scores of participants and provide
a final sorted list for all participants.

\begin{enumerate}

\item For each university, how do they sort the scores of participants
  belonging to it? Please briefly describe a comparative sorting
  algorithm that is appropriate for this purpose and a non-comparative
  sorting algorithm that works in this setup.

\item How does the organizer sort the scores of participants given $k$
  files, where each file includes the sorted scores of participants
  from a specific university?  Please describe an algorithm with a
  $O(n\log k)$ running time and justify its time complexity.

\item Suppose the organizer want to figure out the participants of
  ranking $r$. Given $k$ sorted files, how does the organizer find the
  $r$th largest scores without sorting the scores of participants? Please 
  describe an algorithm in $O(k(\sum_{i=1}^{k}\log m_{i}))$ time and justify
  its time complexity. 

Could you do this in $O(k+\log k \sum_{i=1}^{n}\log m_{i})$ time? 

  [Hint: If $r$ is larger than $\dfrac{n}{2}$, the elements that have at least $\dfrac{n}{2}$ 
  elements larger than them should not be possible answer. The problem is how to identify these
  elements.]\\

\end{enumerate}


\noindent {\bf Problem 4:} You have a collection of $n$ New York Times
crossword puzzles from 01/01/1943 until 12/31/2012 stored in a
database. The only operations that you can perform to the database are
the following:
\begin{itemize}
\item crossword\_puzzle $x$ $\leftarrow$ getPuzzle( int index ); where the
  index is between $1$ and $n$; the puzzles are \emph{not} sorted in the
  database in terms of the date they appeared.
\item getDay( crossword\_puzzle x ); which returns a number between 1
  to 31.
\item getMonth( crossword\_puzzle x ); which returns a number between
  1 to 12.
\item getYear( crossword\_puzzle x ); which returns a number between
  1943 to 2012.
\end{itemize}
All of the above queries can be performed in constant time. You have
found out that the number of puzzles is less than the number of days
in the above period (from 01/01/1943 until 12/31/2012) by one, i.e.,
one crossword puzzle was not included in the database. We need to
identify the date of the missing crossword puzzle.

Design a linear-time algorithm that minimizes the amount of space that
it is using to find the missing date. Ignore the effect of leap years.\\

\begin{enumerate}

\item Start by creating an array of integers (all inititilized to 0) with a length of the number of years (1943 - 2012).

\item Go through the entries in the database using getPuzzle() starting at zero, and perform
a getYear() on all of them.  For every puzzle, go to the array index corresponding to that puzzles 
year and increase it by 1. 

\item Traverse the array and find the year (array index) that doesn't have a value of 365.  This is the year of the 
missing puzzle.

\item Create another array of length 12 (number of months) and inititialize each index to 0.

\item Go through the entries in the database again and do a getYear on them.  If the year matches the 
year of the missing puzzle, do a getMonth() on it.  Then go to the month array index corresponding to the month
and increment the value at that index.

\item After that, go through the month array and see which month has one less day than it should have.  This is the month of the missing puzzle.

\item Create another array with a length matching the number of days in the missing month
and inititialize each index to 0.

\item Go through the entries in the database again and do a getYear() on them.  If the year matches the 
year of the missing puzzle, do a getMonth() on it.  If the month matches the month of the missing puzzle, do a getDay()
on it.  Then go to that index of the array of days and increment the value.  

\item Then go through the day array and find which index has a value of 0.  This is the day of the missing puzzle.

\item Now you have the date of the missing puzzle.

\end{enumerate}

\begin{center}
{\bf Part D (25 points)}
\end{center}

\noindent {\bf Problem 5:} You are running a promotional event for a
company during which the plan is to distribute $n$ gifts to the
participants. Consider that each gift $i$ is worth an integer number
of dollars $a_i$. There are $m$ people participating in the event,
where $m < n$. The $j$-th person is satisfied if he receives gifts
that are worth at least $s_j$ dollars each. The task is to satisfy as
many people as possible given that you have a knowledge of the gift
amounts $a_i$ and the satisfaction requirements of each person
$s_j$. Give an approximation algorithm for assigning rewards to people
with a running time of $O(m\log m+n)$. What is the approximation ratio
of your algorithm and why?\\

\noindent Let $n$ be the number of gifts and $m$ be the number of people participating in the event, where $m<n$. To satisfy as many people as possible given the gift amounts $a_i$ and the satisfaction requirements of each person $s_j$, we will sort gifts in terms of their prices from lowest price to highest price, using radix sort. Because there are $n$ gifts, and radix sort is linear, we have $n$ as our running time now. To continue, we will build a min heap with the people's satisfaction requirements, and building a min heap is linear time, so we have a running time of $m$ in addition to the $n$ from sorting the gifts. To proceed, we will take the minimum satisfaction requirement from the heap, which requires a worst case running time of $log(m)$ to traverse the heap. We will take the minimum satisfaction requirement and compare it to first value -- smallest value -- in sorted gifts. If that gift value is greater than or equal to that person's satisfaction requirement, then we remove that person from the heap, and we remove the gift from the list. If that gift value is not greater than or equal to that person's satisfaction requirement, then we continue traversing the sorted gifts, and whenever we hit a gift that satisfies the person, then we remove the person from the heap and the gift from the list. Continue removing the minimum value from the heap until each person gets a gift or until no gift satisfies that person. Therefore, this algorithm will satisfy as many people as possible with the running time of $O(mlogm + n)$.


\end{document}

