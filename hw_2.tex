\documentclass{article}
\usepackage{amssymb}
\usepackage{amsmath}
\usepackage{amsfonts}
\usepackage{latexsym}
\usepackage{times}
%\usepackage{psfrag,psfig,epsfig,epsf}
\usepackage{graphics}
\usepackage{multirow}
\usepackage{fullpage}
\usepackage{verbatim}
\usepackage{fancyheadings}
\usepackage[T1]{fontenc}
\usepackage{arev}
\usepackage{subfigure}
\usepackage{url}
\usepackage[noline,noend,ruled,linesnumbered]{algorithm2e}
\usepackage{algpseudocode}
\linespread{1.02} 

\pagestyle{empty}

\addtolength{\topmargin}{-20pt}
\addtolength{\oddsidemargin}{-5pt}
\addtolength{\textwidth}{20pt}
\addtolength{\textheight}{50pt}

\newenvironment{myitem}{\begin{list}{$\bullet$}
{\setlength{\itemsep}{-0pt}
\setlength{\topsep}{0pt}
\setlength{\labelwidth}{0pt}
%\setlength{\labelsep}{0pt}
\setlength{\leftmargin}{10pt}
\setlength{\parsep}{-0pt}
\setlength{\itemsep}{0pt}
\setlength{\partopsep}{0pt}}}%
{\end{list}}

\begin{document}

\sloppy

\noindent \underline{CS 344: DESIGN AND ANALYSIS OF COMPUTER
  ALGORITHMS \hspace{1.6in} SPRING 2014}

\vspace{0.1in}

\begin{center}
{\bf {\large Homework 2}}\\
Divide-and-Conquer Algorithms, Sorting Algorithms, Greedy Algorithms\\
\end{center}

\vspace{0.1in}

\noindent Deadline: March 7, 11:59pm.\\ 
Available points: 110. Perfect score: 100.\\

\begin{center}
{\bf Homework Instructions:}
\end{center}

\noindent {\bf Teams:} Homeworks should be completed by teams of
students - three at most. No additional credit will be given for
students that complete a homework individually. Please inform
Athanasios Krontiris about the members of your team as soon as
possible - even if your ``team'' is just a single person (email: ak979
AT cs.rutgers.edu).\\

\noindent {\bf Submission Rules:} Submit your solutions electronically
as a PDF document through \url{sakai.rutgers.edu}. Do not submit Word
documents, raw text, or hardcopies etc. Make sure to generate and
submit a PDF instead. Each team of students should submit only a
single copy of their solutions and indicate all team members on their
submission.  Failure to follow these rules will result in lower grade
in the assignment.\\

\noindent {\bf Late Submissions:} No late submission is allowed. If
you don't submit a homework on time, you get 0 points for that
homework.\\

\noindent {\bf Extra Credit for \LaTeX:} You will receive 10\% extra
credit points if you submit your answers as a typeset PDF (using
\LaTeX, in which case you should also submit electronically your
source code). Resources on how to use \LaTeX\ are available on the
course's website. There will be a 5\% bonus for electronically
prepared answers (e.g., on MS Word, etc.) that are not typeset.\\

\noindent {\bf 25\% Rule:} For any homework problem (same will hold
for exam questions), you can either attempt to answer the question, in
which case you will receive between 0 and 100\% credit for that
question, or you can write "I don't know", in which case you receive
25\% credit for that question. Leaving the question blank is the same
as writing "I don't know." You can and will get less than 25\% credit
for a question that you answer erroneously.\\

\noindent {\bf Handwritten Reports:} If you want to submit a
handwritten report, scan it and submit a PDF via Sakai. We will not
accept hardcopies. If you choose to submit handwritten answers and we
are not able to read them, you will not be awarded any points for the
part of the solution that is unreadable.\\

\noindent {\bf Precision:} Try to be precise. Have in mind that you
are trying to convince a very skeptical reader (and computer
scientists are the worst kind...) that your answers are correct.\\

\noindent {\bf Collusion, Plagiarism, etc.:} Each team of students
must prepare its solutions independently from other teams, i.e.,
without using common notes or worksheets with other students or trying
to solve problems in collaboration with other teams.  You must
indicate any external sources you have used in the preparation of your
solution. Do not plagiarize online sources and in general make sure
you do not violate any of the academic standards of the course, the
department or of the university (the standards are available through
the course's website). Failure to follow these rules may result in
failure in the course.\\

\newpage

\vspace{0.1in}

{\bf }

\begin{center}
{\bf Part A (20 points)}
\end{center}

\noindent {\bf Problem 1:} The more general version of the Master
Theorem is the following. Given a recurrence of the form: 
$$T(n) = a T(\frac{n}{b}) + f(n)$$
where $a \geq 1$ and $b > 1$ are constants and $f(n)$ is an
asymptotically positive function, there are 3 cases: 
\begin{enumerate}
\item If $f(n) = O(n^{log_ba - \epsilon}$) for some constant $\epsilon
  > 0$, then $T(n) = \Theta(n^{log_ba})$.
\item If $f(n) = \Theta(n^{log_ba} log^kn)$ with $k \geq 0$, then
  $T(n) = \Theta(n^{log_ba} log^{k+1}n)$. In most cases, $k = 0$.
\item If $f(n) = \Omega(n^{log_ba+\epsilon})$ with $\epsilon > 0$, and
  $f(n)$ satisfies the regularity condition, then $T(n) = \Theta( f(n)
  )$. The regularity condition specifies that $a f(\frac{n}{b}) \leq c
  f(n)$ for some constant $c < 1$ and all sufficiently large $n$.
\end{enumerate}

\noindent Give asymptotic bounds for the following recurrences. Assume
$T(n)$ is constant for $n = 1$. Make your bounds as tight as possible,
and justify your answers.\\

\noindent A. $T(n) = 2T(\frac{n}{4}) + n^{0.51}$\\
\noindent B. $T(n) = 16 T(\frac{n}{4}) + n!$\\
\noindent C. $T(n) = \sqrt{2} T(\frac{n}{2}) + logn$\\
\noindent D. $T(n) = T(n-1) + lgn$\\
\noindent E. $T(n) = 5T(n/5) + \frac{n}{lgn}$\\

\begin{center}
{\bf Part B (25 points)}
\end{center}

\noindent {\bf Problem 2:} You are in the HR department of a company,
and here is a job for you.  There are $n$ different projects, and $n$
different programmers.  Every project has its unique bonus and
difficulty.  Every programmer has his unique capability and
expectation of bonus.

You can arrange a meeting between each project manager and programmer.
In each meeting, the project manager will give the programmer an
interview to see whether the programmer can do the project; the
programmer can ask the project manager about the bonus to see whether
the bonus can meet his expectation.  After a meeting, you can get a
result based on the feedback of the project manager and the
programmer.  The result can be:

\begin{enumerate}
  \item The programmer can't do the project.
  \item The programmer can do the project, but the bonus of the
    project doesn't meet his expectation.
  \item The programmer can do the project, and the bonus of the
    project meets his expectation. At this time, we say the project
    and the programmer \textit{match} with each other.
\end{enumerate}

Due to the limitation of your authority and information, you can't
compare the bonus or difficulty of two projects, or the capability or
expectation of bonus of two programmers.

Assume that the projects and programmers match one to one. Your goal
is to match each programmer to a project.

{\bf A.} Show that any algorithm for this problem must need $\Omega( n
\log{n} )$ meetings in the worst case.

{\bf B.}  Design a randomized algorithm for this problem that runs in
expected time $O ( n \log{n} )$.

\newpage

\begin{center}
{\bf Part C (40 points)}
\end{center}


\noindent {\bf Problem 3:} A national wide programming contest is hold
at $k$ universities in North America. The $i$th university has $m_{i}$
participants. The total number of participants is $n$, i.e., 
$n = \sum_{i=1}^{k}m_{i}$.In the contest, participants have to write 
programs to solve 6 problems. Each problem contain 10 test cases, each test case
worth 10 points. Participants have to make their program to pass the
test cases as much as possible. After the contest, each university sorts
the scores of participants belonging to it and submit it to the organizer.
Then the organizer has to collect the sorted scores of participants and sort 
them in order. 

\begin{enumerate}

\item For each university, how do they sort the scores of participants
  belonging to it? Please briefly describe one comparative sorting
  algorithm and one non-comparative sorting algorithm.

\item How does the organizer sort the scores of participants given $k$
  files including the sorted scores of participants? Pleas describe an
  algorithm in $O(n\log k)$ time and justify the time complexity.

\item Suppose the organizer want to figure out the participants of
  ranking $r$. Given $k$ sorted files, how does the organizer find the
  $r$th largest scores without sorting the scores of participants? Please 
  describe an algorithm in $O(k(\sum_{i=1}^{k}\log m_{i}))$ time and justify
  its time complexity. Could you do this in $O(k+\log k \sum_{i=1}^{n}\log m_{i})$ time? 
  [Hint: If $r$ is larger than $\dfrac{n}{2}$, the elements that have at least $\dfrac{n}{2}$ 
  elements larger than them should not be possible answer. The problem is how to identify these
  elements.]\\
  

\end{enumerate}


\noindent {\bf Problem 4:} You have list of daily records stored in
list $A$. There should be one record for every day between two exact
days. Yet, we found that the number of records is less than the number
of days by one. We need to identify the date of the missing
record. You don't have access to the entire date of the record. For
each record $i$, you can use one of the following three functions at a
time: getDay(), getMonth() or getYear().

Design an efficient algorithm (w.r.t time and space) to find the missing date. You are free to put any reasonable assumption. [Notes: the time complexity for this problem should be linear in number of records. Part of the grade will be for minimizing the extra space used]


\begin{center}
{\bf Part D (25 points)}
\end{center}

\noindent {\bf Problem 5:} You have $n$ rewards, each reward grants
integer number of points. The $i$-th reward gives $a_i$ points. You
have $m$ persons, where $m < n$. The $j$-th person ($P_j$) is
satisfied if he got total of $s_j$ points or more. The task is to
satisfy as many persons as possible. Give an efficient approximation
algorithm for assigning rewards to persons. Show the approximation ratio of your algorithm. 
[Note: The best possible ratio is to fill at least $1/2$ as many persons as the
optimal solution. To achieve this ratio you can use algorithm runs in $O(m\log m+n)$]


\end{document}

